\documentclass[aspectratio=169, sectionpage=false, german]{tumbeamer}
\usepackage{tumuser}
\usepackage{tumcolors}
\usepackage[utf8]{inputenc}
\usepackage[autostyle]{csquotes}
\usepackage[T1]{fontenc}
\usepackage{xstring}
\usepackage{catchfile}
\usepackage{amsmath}
\usepackage{amsfonts}
\usepackage{amssymb}
\usepackage{float}
\usepackage{graphicx}
\usepackage{booktabs}
\usepackage{icomma}
\usepackage{siunitx}
\usepackage{multicol}
\usepackage[german]{varioref}
\usepackage{listings}
\usepackage{color}
\usepackage{tikz}
\usepackage{pgfplots}
\usepackage{pgfplotstable}

% environment setup for pgfplots and tikz
\pgfplotsset{compat=newest}
\definecolor{era-qemu}{HTML}{FFB74D}
\definecolor{era-dbt-1}{HTML}{f44336}
\definecolor{era-dbt-2}{HTML}{64B5F6}
\definecolor{era-native}{HTML}{AED581}
\definecolor{era-dbt-3}{HTML}{ba00e5}

\newcommand{\blueverb}[1]{{\color{TUMBlue} \texttt{#1}}}

\TUMbeamersetup{
title page = TUM tower,
part page = TUM toc,
section page = TUM blue,
content page = TUM more space,
headline = TUM threeliner,
footline = TUM infoline,
headline on = {title page, section page},
footline on = {section page, content page, part page=false},
}

% setup commands
\newcommand*\diff{\mathop{}\!\mathrm{d}}
\newcommand*\Diff[1]{\mathop{}\!\mathrm{d^#1}}
\newcommand{\define}[2]{\item \textbf{#1}\\#2}
\newcommand{\conclude}[0]{\ensuremath{\Longrightarrow} }
\newcommand{\refer}[0]{\ensuremath{\rightarrow} }
\sisetup{range-phrase=--,range-units=single}
\MakeOuterQuote{"}

% config for source code listings
\lstset{
	language=C++,
	aboveskip=3mm,
	belowskip=3mm,
	showstringspaces=false,
	columns=flexible,
	basicstyle={\ttfamily},
	breaklines=true,
	breakatwhitespace=true,
	tabsize=3
}

% document properties
\title[Binary Translation: RISC--V \refer x86--64]{Dynamische Binärübersetzung:\\RISC--V \refer x86--64}
\subtitle{Projektvorstellung}
\author[Dormann, Kammermeier, Pfannschmidt, Schmidt]{Noah Dormann\inst{1}, Simon Kammermeier\inst{1},\\Johannes Pfannschmidt\inst{1}, Florian Schmidt\inst{1}}

\institute[]{\inst{1} Fakultät für Informatik,
Technische Universität München (TUM)}
\date{\today}

% todo remove: typeset current commit id for draft management
% \CatchFileDef{\headfull}{../../.git/HEAD}{}
%\StrGobbleRight{\headfull}{1}[\head]
%\StrBehind[2]{\head}{/}[\branch]
%\CatchFileDef{\commit}{../../.git/refs/heads/\branch}{}
%\footline{Draft after commit \commit on branch \branch.}

% beamer setup todo uncomment footer after removing commit id
\footline{\insertshortauthor~|~\insertshorttitle}
\addtoTUMcolortheme{TUM default}{
	\setbeamercolor{subtitle}{fg=TUMBlack}
	\setbeamercolor{framesubtitle}{fg=TUMBlack}
}

\begin{document}
\maketitle

\begin{frame}
	\frametitle{Gliederung}
	\tableofcontents
\end{frame}

\section{Motivation}
\begin{frame}[c]
	\frametitle{Motivation}
	\centering\Huge
	Warum Großpraktikum?
\end{frame}


\begin{frame}
	\frametitle{Motivation}
	\framesubtitle{Warum Großpraktikum?}
	
	Arbeitsaufwand bedeutend höher als beim ERA-P.
	\pause
	\textbf{Aber:}
	
	\vspace{0.25cm}
	
	\begin{block}{Lessons learned}
		\begin{itemize}
			\item tiefgreifendes Einarbeiten in ein komplexes Thema
			\item langfristige Arbeit an einem großen Softwareprojekt
			\item viel Erfahrung mit Git ($\sim$ 1000 commits)
			\item Aufrechterhalten einer großen Codebase ($\sim$ 19.000 LOC)
			\item Benchmarking und Testing mit professionellen Tools/Hardware
		\end{itemize}
	\end{block}
\end{frame}

\begin{frame}
	\frametitle{Motivation}
	\framesubtitle{Warum Großpraktikum?}
	
	Zudem:
	
	\vspace{0.25cm}
	
	\begin{block}{Soft Skills}
		\begin{itemize}
			\item Teambuilding und kollaborative Aufgabenverteilung
			\item Projektmanagement und langfristige Zeitplanung
			\item Wissenschaftliches Schreiben
		\end{itemize}
	\end{block}
\end{frame}

\begin{frame}
	\frametitle{Motivation}
	\framesubtitle{Warum Großpraktikum?}
	
	Sehr angenehm:
	
	\vspace{0.25cm}
	
	\begin{block}{Vorteile im Großpraktikum}
		\begin{itemize}
			\item \textbf{bedeutend freiere} Aufgabenstellung
			\item persönliche \textbf{Betreuung} und Feedback am Lehrstuhl
			\item \textit{Bragging rights}
		\end{itemize}
	\end{block}
\end{frame}


\section{Demo}
\begin{frame}[c]
	\frametitle{Demo}
	\framesubtitle{But can it run Crysis?}
	\centering\Huge\ttfamily
	./translator
\end{frame}


\section{Kontakt}
\begin{frame}[c]
	\frametitle{Kontakt}
	\textbf{Repository:}\hspace{0.25cm} {\color{TUMBlue}\url{https://github.com/ria-jit/ria-jit}}
	
	\vspace{0.75cm}
	
	\begin{itemize}
		\item Noah Dormann\\\blueverb{n.dormann@tum.de}
		\item Simon Kammermeier\\\blueverb{simon.kammermeier@tum.de}
		\item Johannes Pfannschmidt\\\blueverb{johannes.pfannschmidt@tum.de}
		\item Florian Schmidt\\\blueverb{fs.schmidt@tum.de}
	\end{itemize}
\end{frame}



\end{document}
